\documentclass[french,12pt, a4paper,twoside,openright]{report}
\usepackage[T1]{fontenc}
\usepackage[french, ruled, vlined]{algorithm2e}
\usepackage{graphicx}
\usepackage{makeidx}
\usepackage{listings}
\usepackage{color}
\usepackage[utf8]{inputenc}
\usepackage{geometry}
\geometry{verbose,tmargin=2cm,bmargin=2cm,lmargin=2cm,rmargin=2cm,headsep=1cm}

\makeglossary

\begin{document}

\title{Développement de web services 
d’import de données pour la constitution automatique 
de réseaux de transport multi-modaux}
\author{Bertrand \textsc{GUERRERO}}
\date{Le \today}


\maketitle                      % Pour produire la page de titre.

\strut\thispagestyle{empty}     % Je mets un truc invisible (\strut) sur la
                                % page juste apres. On met les references ISBN
                                % dans les bouquins. et j'indique que je ne souhaite
                                % pas voir le numero de page (\this...{empty})
\vfill                          % Je remplis avec du rien (\vfill}
\pagebreak                      % Je change de page.

\setcounter{page}{1}            % Je recommence la numerotation a un pour pas
                                % decaler l'alternance droite/gauche et que
                                % les pages impaires soient bien a droite.

\tableofcontents

\chapter*{Résumé}
\addcontentsline{toc}{chapter}{Résumé}
\chaptermark{Résumé}

Voici l'introduction de mon rapport, elle est tr\`es jolie et tout et tout.
Tu noteras que dans un rapport en \LaTeX{}, il n'y a que ce qui se trouve
\textsc{avant} le \verb"\begin{document}" qui est d\'ependant de la machine
utilis\'ee.

En effet, le reste est du standard \LaTeX{} \`a peu pr\`es ind\'ependant de
l'installation effectu\'ee. Par exemple, il est probable que sur ton site en
Grande-Bretagne le fran\c{c}ais soit convenablement install\'e.

De toutes fa\c{c}on tu t'en fiche, je pense que ton rapport sera en anglais.
Pour ce faire, il suffit de virer le \verb"[french]" dans la premiere ligne,
le \verb"\usepackage[T1]{fontenc}" et le \verb"babel". Apr\`es, tu te retrouves
avec le standard am\'ericain. L\`a o\`u cela se corse c'est que si tu met du
fran\c{c}ais, il risque de faire des fautes. Par exemple, avec les mod\`eles
de c\'esure\footnote{coupure des mots en fin de ligne} am\'ericain, il tol\`ere
une c\'esure entre le <<n>> et le <<c>> de <<donc>>. Si << et >> donnent des
points d'interrogation bizarres, c'est que tu as vir\'e la ligne sur le
\verb"fontenc".

\chapter*{Abstract}
\addcontentsline{toc}{chapter}{Abstract}
\chaptermark{Abstract}

\part{\'Etude pr\'eliminaire}

\chapter{On a commenc\'e}

\section{Le commencement}

C'est par l\`a qu'on a commenc\'e, comme d'habitude.

\section{Juste apr\`es}

Ben on a continu\'e de commencer. Mais plus fort.

\section{Enfin}

On a fini de commencer, ce qui fait qu'on \'etait \'epuis\'e. C'est harassant de
devoir travailler des heures enti\`eres (oui, facilement deux depuis le d\'ebut).
Et comme on s'emmerdait, on a pris le temps de lire des livres passionnant sur
le sujet, alors on a mis une citation rigolote:

\begin{quote}\small
As promised in the first edition of this book, C++ has been evolving to meet
the needs of its users. This evolution has been guided by the experience of
users of widely varying backgrounds working in a great range of application
areas. The C++ user-community has grown a hundredflod during the six years
since the first edition of this book; many lessons have been learned, and
many techniques have been discovered and/or validated by experience. Some of
these experiences are reflected here.

\hfill \textit{The C++ Programming Language}, Bjarne \textsc{Stroustrup}
\end{quote}

\chapter{On a \'etudi\'e pr\'eliminairement}

Y fallait bien qu'on s'occupe.

\section{Pr\'eliminaires \`a l'\'etude pr\'eliminaire}

Ben, euh\ldots, dodo, p'tit dej' Kfet, re-dodo, mais en cours cette fois,
pis ensuite manger-pas-bon-Eurest, pis c'est tout. On allait pas bosser
non plus. Non mais.

\section{Pr\'eliminaires \`a l'\'etude principale}

Ben, pile poil pareil.

\section{\'Etude pr\'eliminaire finale}

Pfou\ldots{} Affreux. Tu peux pas t'imaginer. \c{C}a a demand\'e au moins
dix minutes de boulot pour se souvenir ou on avait ranger le jeux de belote.

\chapter{On a retenur de l'\'etude pr\'eliminaire qu'on a fait \`a la Kfet}

\section{Le jeu de belote il est tout sale}

Alors il faut en acheter un autre tout neuf pour qu'on puisse faire le difficile
projet!

\section{La Kfet elle ferme tr\`es t\^ot}

Alors on pourra pas bosser beaucoup vu qu'on arrive a midi.

\section{Eurest c'est pas bon}

Alors on ira manger ailleurs autant qu'on pourra.

\part{\'Etude principale}

\chapter{Le premier sur l'\'etude principale}

\chapter{Le second sur l'\'etude principale}

\chapter{Le troisi\`eme sur la m\^eme chose}

\part{Les projets d'avenir}

\chapter{Extension de la Kfet \`a tout l'ESIEE}

\chapter{Suppression des cours d\'ebiles}

\chapter{Euthanasie des profs cons}

\chapter{Euthanasie des administratifs cons}

\section{Y sont nombreux en plus}

\chapter*{Conclusion}
\addcontentsline{toc}{chapter}{Conclusion}
\chaptermark{Conclusion}

Heureusement que le projet il durait pas plus longtemps parce que \c{c}a coute
vachement cher de glander \`a la Kfet toute la journ\'ee.

\part{Annexes}

\appendix

\chapter{R\'esultats chiffr\'es}

\begin{tabular}{|l|r|r|r|}                                        \hline
                      & D\'epenses     & Recettes   & Bilan    \\ \hline
Caf\'e                & 3658,25 F      & 1,75 F     & -3656,50 \\ \hline
Croissant             & 1548,12 F      & 0,00 F     & -1548,12 \\ \hline
Beignet               &  925,30 F      & 8,25 F     &  -917,05 \\ \hline
                                                                  \hline
Total                 & 6131,67 F      & 10,00 F    & \textbf{-6121,67} \\ \hline
\end{tabular}

\chapter{R\'esultats humains}

Ben je m'est beaucoup amus\'e. Pas vous?

\end{document}
