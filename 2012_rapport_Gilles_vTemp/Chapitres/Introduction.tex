\chapter{Introduction} 
\label{Introduction}

Le stage s'inscrit dans la formation du Master 2 Géomatique et s'est déroulé au sein de deux équipes de l'UMR Espace-Dev \footnote{http://www.espace.ird.fr/index.php}: l'équipe SIC et l'équipe AIMS.\\
 
Les objectifs de l'UMR Espace-Dev sont multiples. L'UMR s'inscrit dans une perspective de développement durable des territoires et proposes des méthodologies de spatialisation des dynamiques de l'environnement. L'UMR développe et exploite également un réseau de stations de réception d'images satellites d'observation de la terre.\\

L'UMR se regroupe en trois équipes de recherche: \\
\begin{itemize}
\item Equipe OSE (Observation spatiale de l'environnement), spécialisée dans la télédétection et les images satellitaires
\item Equipe AIMS (Approche intégrée des milieux et des sociétés) qui est spécialisée dans le domaine de l'environnement, de la  télédétection et dans l'élaboration de dynamiques socio-environnementales ou d'indicateurs et de modèles des interactions milieux/sociétés.
\item Equipe SIC (Systèmes d'information et de connaissances) qui a pour objectif l'acquisition, la gestion, la représentation et le partage des données et des connaissances. D'autres objectifs de l'équipe sont la modélisation de dynamiques spatio-temporelles, la visualisation, la cartographie sémantique et l'aide à la décision\\

\end{itemize}

Tout scientifique qui travaille à établir des cartes de risque fonde ses recherches sur un raisonnement qui consiste à enchaîner des procédures, traitements ou algorithmes. \\

Depuis plusieurs années  au sein de l'UMR Espace-Dev (équipes SIC et AIMS) est développé le SIEL (Système d'Information sur l'Environnement l'Echelle Locale). Le SIEL est un logiciel d'aide à la décision dans la gestion de l'environnement. Il permet notamment d'évaluer le risque de dégradation de la végétation en milieu aride et de produire des \textbf{indices environnementaux spatialisés} sous la forme de cartes. L'objectif du SIEL est de mettre à la disposition des scientifiques des outils informatiques pour automatiser au mieux leur démarche.\\

L'objectif du présent stage est d'élargir les potentialités du SIEL en définissant dans un premier temps un nouveau contexte, celui dénommé environnement-santé et plus précisément celui relatif au paludisme. Le développement d'un outil autorisant la mise en \oe uvre de chaînes de traitements dédiées à l'évaluation de risques environnementaux est donc un des objectifs poursuivis. Basé sur une plateforme Open Source, l'outil devrait permettre à long terme un fonctionnement plus ouvert, plus facile et plus adapté à diverses problématiques. Cet outil sera aussi éprouvé et validé dans le contexte environnement-santé pour ce stage. \\


Le mémoire s'organise de la façon suivante:\newline

\begin{itemize}
\item Une première partie sera dédiée à la présentation du contexte du stage. Dans cette partie sera présenté plus en détail le SIEL ainsi que les problématiques liées à la définition des facteurs de risque, à la construction des indicateurs et au développement d'une chaîne de traitements permettant de les cartographier.\\

\item Dans la deuxième partie, le contexte de la thématique de recherche sous-jacente au stage sera expliqué. A partir d'une recherche bibliographique les principes généraux du thème  environnement-santé ainsi que les approches logicielles existantes relatives à l'enchaînement de traitements et à la cartographie des indicateurs de risque seront présentés.\\

\item Une troisième partie présentera la méthodologie de mon travail. De manière classique, nous commencerons par une phase d'analyse. Celle-ci sera dédiée à la compréhension des éléments liés au cycle du paludisme. A partir du modèle conceptuel de ces élements, l'analyse des experts sera formalisée sous forme de chaînes de traitements préfigurant d'une part leur raisonnement et d'autre part donnera les spécifications  nécessaires pour l'élaboration et le développement du prototype. Les différentes étapes de l'élaboration de la chaîne de traitements seront proposées en commençant avec la  conceptualisation des traitements et des données utilisées pour arriver à l'automatisation de la chaîne de traitements automatisée "fermée" et du logiciel ouvert.\\

\item Dans la dernière partie du mémoire seront présentés les résultats, les difficultés rencontrées et les perspectives de mon travail.\newline

\end{itemize}

Par la suite, les termes en \textbf{gras} seront définis dans le glossaire en fin du mémoire.

%Dans un premier temps, la compréhension de la problématique (environnement-santé / paludisme) est indispensable pour la conceptualisation et de l'élaboration des outils de traitements de risque. Dans cette partie du travail, un modèle conceptuel du cycle de la malaria est proposé à partir des facteurs de risque de cette maladie. A partir de ce modèle, un raisonnement sur les données et les traitements permet de définir les données et les traitements nécessaires pour l'élaboration et le développement de la chaîne de traitements respectivement du logiciel ouvert. Le développement informatique peut être regroupé en deux parties: Premièrement, le développement d'une chaîne de traitements automatisée qui représente sous forme de cartes le risque de transmission du paludisme et deuxièmement le développement d'un logiciel ouvert permettant d'exécuter l'ensemble des traitements de la chaîne selon les besoins de l'utilisateur et d'adapter les traitements à de nouvelles problématiques.
%
%Présenter l'UMR et ses objectifs etc.
%Prélambule, faire comprendre aux gens ce que je fais, je définis les objectifs.
%Le mémoire est structuré en plusieurs chapitres...




