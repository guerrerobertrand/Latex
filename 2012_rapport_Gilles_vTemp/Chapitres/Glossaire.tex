\chapter{Glossaire et définitions}


\textbf{Diagramme de classes} : Un diagramme de classes fournit une vue globale d'un système en présentant ses classes, interfaces et collaborations, et les relations entre elles. Les diagrammes de classes sont statiques : ils affichent ce qui interagit mais pas ce qui se passe pendant l'interaction \footnote{\url{http://docwiki.embarcadero.com/RADStudio/fr/D\%C3\%A9finition_des_diagrammes_de_classes_UML_1.5}} \\

\textbf{Données vecteur / Fichiers de forme / Données shape : } Le mode vecteur où chaque objet représenté sur la carte est décrit par des points successifs composant son pourtour. Chaque point est localisé par ses coordonnées rectangulaires et est joint au point suivant par un segment de droite (d'où le terme de vecteur).\\

Le mode vecteur ne peut s'appliquer qu'à une carte. Le mode raster peut s'appliquer indifféremment à une carte ou à une image. On peut convertir des données raster en données vecteur (vectoriser), ou convertir des données vecteurs en pixels (pixeliser ou rasteriser) \footnote{\url{http://seig.ensg.ign.fr/fichchap.php?NOCONT=CONT3&NOCHEM=CHEMS001&NOFICHE=FP1&NOLISTE=0&N=0&RPHP=&RCO=&RCH=&RF=&RPF=}}. \\. 

\textbf{Eclipse : }
L'environnement de programmation en Java le plus connu est le projet "Eclipse" de la fondation Eclipse. Ce logiciel simplifie la programmation grâce à un certain nombre de raccourcis et notamment grâce à la possibilité d'intégrer de nombreuses extensions. Au fur et à mesure de l'avancement du code, Eclipse compile automatiquement le code et signale les problèmes qu'il détecte. \\

\textbf{GDAL} : GDAL est une bibliothèque de transformation pour les images raster géoréférencées qui est publié sous une licence du type X / Open Source MIT défini par l'Open Source Geospatial Foundation. La bibliothèque OGR fait également partie de GDAL et fournit une fonctionnalité similaire pour des données vectorielles \footnote{\url{http://www.gdal.org/index.html}}. \\

\textbf{Indices environnementaux spatialisés }: Des indices environnementaux spatialisés sont par exemple des indices de risques de désertification et visent à rendre l'information environnementale plus compréhensible, plus simple, plus claire, plus immédiate et plus pertinente. L'approche qui intègre la notion de "risque" a comme objectif d’analyser au mieux l'intervention humaine actuelle et future sur les ressources naturelles face aux conditions climatiques et aux potentialités et écologiques \citep{SIEL2012}. \\

\textbf{Java : }
Java est un langage orienté objet, c'est-à-dire que le programme est vu comme un ensemble d'entités (de classes). Au cours de l'exécution du programme,
les entités collaborent entre elles pour arriver à un but commun.\\


\textbf{JDBC : }

JDBC est une API (Application Programming Interface) java. JDBC est un nom déposé et non un acronyme, en général on lui donne la définition suivante : Java DataBase Connectivity. (http://java.developpez.com/faq/jdbc/...)
Cette API est constituée d'un ensemble d'interfaces et de classes qui permettent l'accès, à partir de programmes java, à des systèmes de gestion de bases de données relationnelles (par exemple PostgreSQL). 

\textbf{Logiciel fermé : } Un logiciel fermé est un logiciel surlequel l'utilisateur ne peut pas influencer le déroulement de l'exécution des traitements. Les logiciels fermés sont également dénommés "boîtes noires".\\


\textbf{Logiciel ouvert }: Un logiciel ouvert est, dans le cadre de ce travail, un logiciel qui permet d'exécuter de façon indépendante des traitements. Un logiciel ouvert est également appelée "boîte blache" (contrairement à un logiciel fermé appelé "boîte noire").\\



\textbf{Moustiques anophèle }:  Moustique de la famille des culicidés, dont la femelle transmet le paludisme \footnote{\url{ http://dictionnaire.reverso.net/francais-definition/anoph\%C3\%A8le }}.\\

\textbf{Niches fondamentales} : Ensemble des ressources potentielles qu’une espèce peut utiliser dans son
milieu lorsque les conditions sont idéales \footnote{\url{http://www.ese.u-psud.fr/epc/conservation/EE/pdf/Ecologie\_Communautes\_15112009.pdf}}.\\

\textbf{PostgreSQL : }

PostgreSQL est un système de gestion de bases de données relationnelles objet (Manuel PostgreSQL). PostgreSQL est un outil Open Source et disponible gratuitement, compatible avec les systèmes d'opérations les plus connus (Linux, Unix (Mac OSX, Solaris etc.) et Windows). PostgreSQL propose des interfaces de programmations pour des langages de programmation comme Java, C++, Python etc.
Le développement de PostgreSQL a débuté en 1986 (appelé à l'époque Postgres). En 1995, les développeurs ajoutent un interpréteur de langage SQL à l'outil. A partir de 1996, l'outil s'appelle PostgreSQL afin de souligner le lien entre Postgres et le langage SQL. PostgreSQL peut être facilement étendu par l'utilisateur en ajoutant de nouvelles fonctions, de nouveaux opérateurs ou même de nouveaux langages de procédure.\\


\textbf{PostGIS : }
PostGIS est une extension du système de gestion de base de données PostgreSQL qui permet de stocker des données (objets) géographiques dans la base de données. Cette extension permet d'utiliser une base de données PostgreSQL comme une base de données dans n'importe quel projet SIG. Depuis avril 2012, la dernière version de PostGIS (PostGIS 2.0) offre de nombreuses améliorations. La nouveauté la plus "innovante" est celle que PostGIS gère désormais les données raster (données images).
PostGIS est compatible avec des nombreux autres outils comme par exemple Mapserver.\\

\textbf{Rastériser} : Conversion de données vectorielles en images raster.  \footnote{\url{http://www.olecorre.com/1298/Rasterization}}.\\

\textbf{Reprojeter : } Changer la projection d'une donnée géoréférencée.\\


\textbf{SGBD} : Système de gestion des de base de données - Un ensemble de programmes qui permettent l'accès à une base de données \footnote{\url{http://www.futura-sciences.com/fr/definition/t/informatique-3/d/sgbd_2525/}}.\\

\textbf{SRID} : Le SRID est un entier qui identifie de façon unique le système de références spatiales.\\


\textbf{Traitement : }

Un traitement est un outil qui permet de transformer une information ou une donnée fournie en entrée et livrée en sortie. Comme nous nous intéressons aux traitements informatiques, nous nous intéresserons principalement à la partie logicielle (software) de ces traitements.
"Un traitement réalise une fonctionnalité. Pour être précis, il faudrait dire qu'un traitement spécifie la façon de réaliser une fonctionnalité. Cette spécification peut être exprimée en langage informatique." (\citep{ElKader2006})

Dans ce mémoire, nous excluons donc une partie des traitements comme la saisie et la création de données (par exemple à partir d'une image satellite). Nous nous intéresserons aux traitements informatiques géographiques nécessaires pour créer une carte du risque.
Les traitements informatiques géographiques regroupent les traitements qui manipulent les données géographiques. 

\textbf{Quickbird} :  Une image Quickbird est une image de très haute résolution. Ce type d'images se caractérisent par une résolution spatiale de moins de 2.5 mètres. Les bâtiments, les rues, les voitures et même les arbres sont visibles sur ces images \footnote{\url{http://www.gim.be/C12574AD00426BEC/_/6626A938C60147B1C12574CC0054F8D3?OpenDocument}}.\\


\textbf{Projection : } Les définitions les plus "techniques" caractérisent la projection cartographique comme étant une méthode mathématique faisant correspondre à un point géoréférencé de l'ellipsoïde terrestre un autre point sur un support en deux dimensions, relativement au système de coordonnées géographiques. Des définitions plus générales présentent la projection comme une méthode géodésique permettant de représenter la Terre sur la surface plane d'une carte \footnote{\url{http://cartographe.servhome.org/projection.php}}.\\ 

\textbf{(Logiciel) R : }

R est un langage et un environnement pour des calculs et des graphiques statistiques qui fournit une grande variété de techniques statistiques. R est Open Source et disponible gratuitement. R ne fonctionne pas sous le principe du "clique bouton" mais doit être compris comme un langage informatique. Un grand nombre d'extensions (packages) est disponible et permet d'utiliser R dans des domaines très divers. Ces extensions permettent également d'appeler R à partir d'autres langages de programmation.\\


\textbf{RCaller : }

RCaller est une extension de Java permettant d'appeler le logiciel R et d'utiliser les fonctionnalités de R dans un programme Java. Cet outil est particulièrement intéressant comme il permet d'appeler les librairies de R comme par exemple rgdal et donc d'utiliser les fonctionnalités de GDAL dans notre chaîne de traitements.\\

\textbf{SIG : }Système informatique permettant, à partir de diverses sources, de rassembler et d'organiser, de gérer, d'analyser et de combiner, d'élaborer et de présenter des informations localisées géographiquement, contribuant notamment à la gestion de l'espace \footnote{\url{www.cartographie.ird.fr/publi/documents/sig1.pdf}}

\textbf{Table dans une base de données : }Dans les bases de données relationnelles, une table est un ensemble de données organisées sous forme d'un tableau où les colonnes correspondent à des catégories d'information (une colonne peut stocker des numéros de téléphone, une autre des noms ou encore la géométrie) et les lignes à des enregistrements footnote{\url{http://fr.wikipedia.org/wiki/Table_\%28base\_de\_donn\%C3\%A9es\%29}}.

\textbf{Taille de cellule Raster} : La taille de la cellule (pixel) détermine le niveau de détail qui peut être représenté dans un raster  \footnote{\url{http://help.arcgis.com/fr/arcgisdesktop/10.0/help/index.html}}.\\


\textbf{Zone tampon / Buffer} : Une zone tampon ou un buffer est une zone à distance fixe autour d'une entité.











