\chapter{Contexte du stage}
\label{AnalyseConception}

\section{Environnement de travail}

Durant le stage j'ai travaillé sur un PC dont le système d'exploitation est Windows 8.1 Profesionnal (machine hôte). L'entreprise travaille avec de nombreuses machines virtuelles hébergées ou distantes afin de disposer d'environnement de tests, de développements, et de production. J'avais donc à ma disposition une machine virtuelle de développement via VirtualBox dont le système d'exploitation était Windows Server 2012 R2 Standard. 
Ma machine était pré-configurée avec tous les outils nécessaires pour développer, (le clone de cette machine virtuelle de développement a pu être utilisé par d'autres stagiaires), ainsi l'administrateur système de MobiGIS maîtrise la configuration logicielle des machines et l'utilisateur perd moins de temps à la configuration.\\

\section{Outils utilisés}

Quotidiennement j'ai utilisé 2 environnements de développement intégré (IDE) : Liclipse (pour développer en langage Python) dans le projet DataWizard, et \textbf{Eclipse} version Mars (pour développer en langage Java/Java EE) pour les projets MobiSAAS et Crislab.\\

Dans les projets plusieurs \textbf{SGBD} sont manipulés : MongoDB (MobiSAAS), Oracle (Crislab), mais j'ai principalement travaillé avec \textbf{Postgresql} afin de gérer les données de mon module. De plus, Postgresql permet de gérer les données géographiques dans le projet \og DataWizard \fg grâce à sa cartouche spatiale \textbf{Postgis}.\\

Les logiciels suivants ont été utilisés quotidiennement : Maven, DropWizard, Hibernate, Jackson, Postgresql, Pgadmin, \textbf{SoapUI}, Java JDK 1.7 et 1.8. Certains de ces outils seront décrits dans la suite de ce mémoire \ref{MobiSAASTechno}.
J'ai également produit des schémas pour la documentation avec des logiciels de « modélisation » comme : ArgoUML, DBVisualizer, Enterprise Architect, yEd.\\

\section{Gestion de projets}

Dans les projets auxquels j'ai participé, l'entreprise utilise des outils de gestion : planning, suivi de bugs, outils de mutualisation, gestion de versions. Ainsi, j'ai utilisé l'outil Trello pour la gestion des tâches, Redmine pour la gestion des projets (cf. Annexe \ref{Annexe A}), et \textbf{SVN} pour la gestion des codes sources. L'entreprise met à disposition de ses salariés un intranet avec de nombreux outils collaboratifs sous la plateforme eGroupware (feuille de temps,etc...).\\

J'ai effectué mon travail en étroite relation avec le chef de projet. A partir des spécifications techniques et fonctionnelles, les développements ont progressé et évolué tout le long de la période de réalisation. Sans pour autant pratiquer une méthode «Agile» au sens strict, plusieurs ajustements (mode itératif) ont été effectués et le dialogue quotidien a permis de toujours rendre mon travail en adéquation aux besoins. \\

Un point de suivi informel était effectué plusieurs fois par semaine avec mon responsable afin de présenter le travail effectué, les résultats intermédiaires, et le travail planifié pour la semaine suivante. Un bilan à la mi-stage a été effectué afin de réajuster les priorités, et arriver à produire un livrable satisfaisant en fin de stage. \\


\section{Difficultés rencontrées}

L'environnement "équipe" et les compétences de chacun ont été propices à les éviter. En effet, j'avais de chaque côté de mon bureau les 2 personnes les plus à même de débloquer des situations, ou de me renseigner sur une question. \\

Pour le projet MobiSAAS, j'ai eu des "difficultés" qui sont communes à chaque évolution de version d'un \textbf{framework}. En effet, au milieu du stage il a fallu "recoder" dans plusieurs parties du code pour s'adapter aux évolutions lors du passage de la version 0.7.1 à la version 0.8.1 de Dropwizard.
Pour le projet DataWizard et d'une manière générale, une des difficultés a été de répondre aux experts et analystes en réseaux de transport et de produire des résultats (Métadonnées) exploitables et conformes à leurs attentes. Cela a demandé du dialogue pour s'accorder sur la terminologie à employer (ex: EdgeType, PTMode,...). Tous ces termes courants pour les experts ne sont pas intuitifs pour les développeurs. \\
