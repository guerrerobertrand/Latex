\chapter{Introduction} 
\label{Introduction}

Ce stage s'inscrit dans le cadre de la formation Concepteur / Développeur délivrée par BGE Haute-Garonne et s'est déroulé durant 3 mois au sein de l'entreprise Mobigis \footnote{http://www.mobigis.fr}: site de l'entreprise.\\
 
Les objectifs de l'UMR Espace-Dev sont multiples. L'UMR s'inscrit dans une perspective de développement durable des territoires et proposes des méthodologies de spatialisation des dynamiques de l'environnement. L'UMR développe et exploite également un réseau de stations de réception d'images satellites d'observation de la terre.\\

L'UMR se regroupe en trois équipes de recherche: \\
\begin{itemize}
\item Equipe OSE (Observation spatiale de l'environnement), spécialisée dans la télédétection et les images satellitaires
\item Equipe AIMS (Approche intégrée des milieux et des sociétés) qui est spécialisée dans le domaine de l'environnement, de la  télédétection et dans l'élaboration de dynamiques socio-environnementales ou d'indicateurs et de modèles des interactions milieux/sociétés.
\item Equipe SIC (Systèmes d'information et de connaissances) qui a pour objectif l'acquisition, la gestion, la représentation et le partage des données et des connaissances. D'autres objectifs de l'équipe sont la modélisation de dynamiques spatio-temporelles, la visualisation, la cartographie sémantique et l'aide à la décision\\

\end{itemize}

Tout scientifique qui travaille à établir des cartes de risque fonde ses recherches sur un raisonnement qui consiste à enchaîner des procédures, traitements ou algorithmes. \\

Depuis plusieurs années  au sein de l'UMR Espace-Dev (équipes SIC et AIMS) est développé le SIEL (Système d'Information sur l'Environnement l'Echelle Locale). Le SIEL est un logiciel d'aide à la décision dans la gestion de l'environnement. Il permet notamment d'évaluer le risque de dégradation de la végétation en milieu aride et de produire des \textbf{indices environnementaux spatialisés} sous la forme de cartes. L'objectif du SIEL est de mettre à la disposition des scientifiques des outils informatiques pour automatiser au mieux leur démarche.\\

L'objectif du présent stage est d'élargir les potentialités du SIEL en définissant dans un premier temps un nouveau contexte, celui dénommé environnement-santé et plus précisément celui relatif au paludisme. Le développement d'un outil autorisant la mise en \oe uvre de chaînes de traitements dédiées à l'évaluation de risques environnementaux est donc un des objectifs poursuivis. Basé sur une plateforme Open Source, l'outil devrait permettre à long terme un fonctionnement plus ouvert, plus facile et plus adapté à diverses problématiques. Cet outil sera aussi éprouvé et validé dans le contexte environnement-santé pour ce stage. \\


Dans ce mémoire sera présenté le travail que j'ai réalisé, cette présentation s'organise de la façon suivante:\newline

\begin{itemize}
\item Une première partie sera dédiée à la présentation du contexte du stage. Dans cette partie sera présenté plus en détail les SIG ...\\

\item Dans la deuxième partie, le contexte du stage sera expliqué. Projets, outils, produits,...\\

\item Une troisième partie présentera la méthodologie de mon travail et des exemples de réalisation...\\

\item Dans la dernière partie de ce mémoire seront présentés un Bilan du stage et un bilan personnel sur la formation.\newline

\end{itemize}

Par la suite, les termes en \textbf{gras} seront définis dans le glossaire en fin du mémoire.

