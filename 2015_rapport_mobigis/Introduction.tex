\chapter{Introduction} 
\label{Introduction}

Ce stage s'inscrit dans le cadre de la formation "Concepteur / Développeur Informatique" délivrée par BGE Haute-Garonne et s'est déroulé durant 3 mois au sein de l'entreprise Mobigis. Les objectifs à l'issue de cette formation sont de savoir concevoir et développer des applications informatiques en utilisant le langage Java/JavaEE, dans un environnement professionnel.\\

Dans ce mémoire sera présenté le travail que j'ai réalisé, avec par exemple les résultats directs de mon action. Mais aussi, les tâches que j'ai effectuées et les moyens utilisés pour les accomplir : langages, frameworks, logiciels,…\\
 
Cette présentation s'organise de la façon suivante:\newline

\begin{itemize}
\item Une première partie sera dédiée à la présentation du contexte du stage. Dans cette partie seront présentés le domaine d'application de ce stage : les Systèmes d'Informations Géographiques (SIG), et l'entreprise MobiGIS.\\

\item Dans la deuxième partie, l'environnement de travail du stage sera expliqué : projets, équipes, outils, produits,...\\

\item Une troisième partie présentera la méthodologie de mon travail (conception/développement) et des exemples de réalisations (codes)...\\

\item Enfin, dans la dernière partie de ce mémoire seront présentés un bilan professionnel et un bilan personnel de cette nouvelle expérience. \\

\item Pour conclure, la liste des compétences du référentiel qui sont couvertes par cette expérience sera présentée.\newline

\end{itemize}

Par la suite, les termes en \textbf{gras} seront définis dans le glossaire en fin du mémoire.

