\chapter*{Résumé}\label{Resume}

Mon sujet de stage porte sur le développement de web services permettant d’exposer des fonctionnalités d’import de données voirie et transport en commun, pour la constitution automatique de réseaux de transport multi-modaux. \\

L'objectif de mon travail est d'exposer des fonctionnalités codées initialement en langage SQL et/ou Python via une API REST (JAX-RS). Afin d'implémenter ces web services REST, j'ai pris en main les chaînes de traitements (Python) et les données à manipuler, découvert le framework orienté micro-services « DropWizard », et intégré le projet « MobiSaas ». J'ai réalisé tous mes développements dans un module Maven inclus dans un projet mutli-modules existant. L'application développée dans ce projet présente une interface d'administration des fonctionnalités (MobiAnalyst en mode SAAS). C'est dans ce cadre que je développe des fonctionnalités d'upload de données sur le serveur. Mon développement (Java EE) a été réalisé en mode « programmation concurrente » afin de supporter plusieurs requêtes simultanées, et rendre ce service asynchrone. Enfin comme perspectives, les classes et méthodes utilitaires que j'ai développées, seront généralisées afin que l'application supporte plusieurs autres formats de données (données vectorielles « Shapefile » par exemple).\\

A l'occasion de ce stage j'ai pu travailler sur plusieurs projets : DataWizard, Crislab, Moveasy, etc... J'ai donc participé à différentes phases du cycle de vie d'un projet de développement logiciel (de la conception (R\&D), au développement de code métier, jusqu'à la livraison au client).\\
