\chapter*{Webographie}
\label{Bibliographie}


\textbf{Sites sur les données/outils \og métiers \fg :}\label{OBA}\\


\url{https://developers.google.com/transit/gtfs/examples/gtfs-feed}\\

\url{https://developers.google.com/transit/gtfs/}\\

\url{https://github.com/google/transitfeed/wiki/FeedValidator}\\

\url{http://onebusaway.org/developer-information/}\\

\url{https://github.com/OneBusAway/onebusaway/wiki/Importing-source-code-into-Eclipse}\\

\url{http://developer.onebusaway.org/modules/onebusaway-gtfs-modules/current/apidocs/index.html}\\


\textbf{Sites sur les frameworks et outils utilisés :}\\


\url{http://www.objis.com/formation-java/tutoriel-formation-maven-2.html}\\

\url{http://mvnrepository.com/}\\

\url{http://www.dropwizard.io/}\\

\url{https://github.com/bucharest-jug/dropwizard-todo}\\

\url{http://blog2dev.blogspot.fr/2015/03/dropwizard.html}\\

\url{http://docs.jboss.org/hibernate/orm/4.3/manual/en-US/html/}\\

\url{http://www.tutorialspoint.com/hibernate/}\\

\url{http://blog.xebia.fr/2011/11/14/rest-java-serveur/}\\

\url{http://blog.xebia.fr/2010/08/18/comparatif-des-librairies-json/}\\


Et pour finir, un grand merci à \url{http://www.mkyong.com/}, tout simplement.
