\chapter{Glossaire et définitions}
\label{Glossaire}


\textbf{Eclipse} : L'environnement de programmation (IDE) en langage Java le plus connu est le projet "Eclipse" de la fondation Eclipse. Ce logiciel simplifie la programmation grâce à un certain nombre de raccourcis et notamment grâce à la possibilité d'intégrer de nombreuses extensions. Au fur et à mesure de l'avancement du code, Eclipse compile automatiquement le code et signale les problèmes qu'il détecte.\\

\textbf{Géomatique} : La géomatique est la combinaison syntaxique de deux mots : Géographie et Informatique.
Le mot géomatique a été déterminé pour regrouper de façon cohérente l’ensemble des connaissances et technologies nécessaires à la production et au traitement des données numériques décrivant le territoire, ses ressources ou tout autre objet ou phénomène ayant une position géographique.
La géomatique est un domaine qui fait appel aux sciences, aux technologies de mesure de la terre ainsi qu’aux technologies de l’information pour faciliter l’acquisition, le traitement et la diffusion des données sur le territoire (aussi appelées "données spatiales ", "données géospatiales" ou " données géographiques").
La géomatique est étroitement liée à l’information géographique qui est la représentation d’un objet ou d’un phénomène localisé dans l’espace.
Ainsi, la géomatique regroupe l’ensemble des outils et méthodes permettant de représenter, d’analyser et d’intégrer des données géographiques
\footnote{\url{http://www.sig-geomatique.fr/sig-geomatique.html}}.\\

\textbf{Java} : Java est un langage orienté objet, c'est-à-dire que le programme est vu comme un ensemble d'entités (de classes). Au cours de l'exécution du programme, les entités collaborent entre elles pour arriver à un but commun.\\

\textbf{PostgreSQL} : PostgreSQL est un système de gestion de bases de données relationnelles objet (Manuel PostgreSQL). PostgreSQL est un outil Open Source et disponible gratuitement, compatible avec les systèmes d'opérations les plus connus (Linux, Unix (Mac OSX, Solaris etc.) et Windows). PostgreSQL propose des interfaces de programmations pour des langages de programmation comme Java, C++, Python etc.
Le développement de PostgreSQL a débuté en 1986 (appelé à l'époque Postgres). En 1995, les développeurs ajoutent un interpréteur de langage SQL à l'outil. A partir de 1996, l'outil s'appelle PostgreSQL afin de souligner le lien entre Postgres et le langage SQL. PostgreSQL peut être facilement étendu par l'utilisateur en ajoutant de nouvelles fonctions, de nouveaux opérateurs ou même de nouveaux langages de procédure.\\

\textbf{PostGIS} : PostGIS est une extension du système de gestion de base de données PostgreSQL qui permet de stocker des données (objets) géographiques dans la base de données. Cette extension permet d'utiliser une base de données PostgreSQL comme une base de données dans n'importe quel projet SIG. PostGIS est compatible avec de nombreux autres outils SIG comme par exemple QGIS, Mapserver, etc...\\