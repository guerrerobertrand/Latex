\chapter{Conclusion}
\label{Conclusion}

Voici la conclusion de mon rapport, elle est tr\`es jolie et tout et tout. Tu noteras que dans un rapport en \LaTeX{}, il n'y a que ce qui se trouve \textsc{avant} le \verb"\begin{document}" qui est d\'ependant de la machine utilis\'ee.

En effet, le reste est du standard \LaTeX{} \`a peu pr\`es ind\'ependant de l'installation effectu\'ee. Par exemple, il est probable que sur ton site en
Grande-Bretagne le fran\c{c}ais soit convenablement install\'e.

De toutes fa\c{c}on tu t'en fiche, je pense que ton rapport sera en anglais. Pour ce faire, il suffit de virer le \verb"[french]" dans la premiere ligne,
le \verb"\usepackage[T1]{fontenc}" et le \verb"babel". Apr\`es, tu te retrouves avec le standard am\'ericain. L\`a o\`u cela se corse c'est que si tu met du
fran\c{c}ais, il risque de faire des fautes. Par exemple, avec les mod\`eles de c\'esure\footnote{coupure des mots en fin de ligne} am\'ericain, il tol\`ere
une c\'esure entre le <<n>> et le <<c>> de <<donc>>. Si << et >> donnent des points d'interrogation bizarres, c'est que tu as vir\'e la ligne sur le
\verb"fontenc".