\chapter{Travail Réalisé}
\label{TravailRealise}

\section{Outils de développement}
\begin{itemize}
\item ECLIPSE:
Eclipse IDE est un environnement de développement intégré libre (le terme Eclipse désigne également le projet correspondant, lancé par IBM) extensible, universel et polyvalent, permettant potentiellement de créer des projets de développement mettant en œuvre n'importe quel langage de  programmation. Eclipse IDE est principalement écrit en Java (à l'aide de la bibliothèque graphique SWT, d'IBM), et ce langage, grâce à des bibliothèques spécifiques, est également utilisé pour écrire des extensions.

\item APACHE TOMCAT :
Tomcat est un conteneur de Servlet J2EE issu du projet Jakarta, Tomcat et est désormais un projet principal de la fondation Apache. C’est un conteneur de Servlet J2EE qui implémente la spécification des Servlets et des JSP de Sun Microsystems. Tomcat est en fait chargé de compiler les pages JSP avec Jasper pour en faire des Servlets (une servlet étant une application Java qui permet de générer dynamiquement des données au sein d’un serveur http). 

\item SVN :
Subversion est un système de gestion de version, conçu pour remplacer CVS. Concrètement, ce système permet aux membres d’une équipe de développeur de modifier le code du projet quasiment en même temps. Le projet est en effet enregistré sur un serveur SVN et à tout moment, le développeur peut mettre à jour une classe avant de faire des modifications pour bénéficier de la dernière version et a la possibilité de comparer deux versions d'un même fichier.

\end{itemize}

\section{Gestion de projet}
Un point de suivi informel était effectué chaque semaine ou deux fois par semaine avec l'encadrant afin de présenter le travail effectué, les résultats intermédiaires, et le travail planifié pour la semaine suivante.
Nous avons attaché aussi une importance à la synthèse des informations relatives aux solutions testées, afin de donner une visibilité à différents niveaux de l'étude :
\begin{itemsize}
\item niveau technique
\item niveau développement
\end{itemsize}